\documentclass[a4paper,12pt]{article}
\usepackage{graphicx}
\usepackage{booktabs}
\usepackage{amsmath}
\usepackage{geometry}


\geometry{margin=1in}

\begin{document}

\title{Correlation Analysis of Life Expectancy Data}
\date{\today}
\maketitle

\section{Introduction}

The aim of this analysis is to identify correlations within a dataset containing life expectancy values for males, females, and the combined population (both sexes) across different years and countries.  
We compute Pearson correlation coefficients and support our findings with appropriate visualisations. The results help us understand how life expectancy evolves over time and how strongly it relates across genders.

\section{Methodology}

We selected all numerical variables in the dataset for correlation analysis:

\begin{itemize}
    \item \textbf{Year}
    \item \textbf{LifeExpectancy\_male}
    \item \textbf{LifeExpectancy\_female}
    \item \textbf{LifeExpectancy\_both}
\end{itemize}

Pearson correlations were computed in Python using pandas, and visualisations were generated using seaborn and matplotlib.  
Three plots were produced to support the interpretation:

\begin{itemize}
    \item Distribution of life expectancy (both sexes)
    \item Scatter plot of male vs female life expectancy
    \item Overlapping distributions of male, female, and both sexes
\end{itemize}

\section{Results}

\subsection{Correlation Matrix}

\begin{table}[h!]
\centering
\begin{tabular}{lcccc}
\toprule
 & Year & Male & Female & Both \\
\midrule
Year & 1.0000 & 0.7672 & 0.7424 & 0.7562 \\
Male & 0.7672 & 1.0000 & 0.9904 & 0.9976 \\
Female & 0.7424 & 0.9904 & 1.0000 & 0.9975 \\
Both & 0.7562 & 0.9976 & 0.9975 & 1.0000 \\
\bottomrule
\end{tabular}
\caption{Pearson correlation coefficients between numerical variables.}
\label{tab:corr}
\end{table}

\subsection{Visualisations}

\begin{figure}[h!]
\centering
\includegraphics[width=0.9\textwidth]{distribution.png}
\caption{Distribution of life expectancy (both sexes, all countries, all years).}
\label{fig:dist_both}
\end{figure}

\begin{figure}[h!]
\centering
\includegraphics[width=0.9\textwidth]{country-year.png}
\caption{Male vs female life expectancy for each country-year. The dashed red line shows the reference $y=x$.}
\label{fig:male_female}
\end{figure}

\begin{figure}[h!]
\centering
\includegraphics[width=0.9\textwidth]{life_expectancy.png}
\caption{Distribution of life expectancy for males, females and both sexes.}
\label{fig:dist_gender}
\end{figure}

\subsection{Interpretation of Results}

Figure~\ref{fig:dist_both} shows the overall distribution of life expectancy for both sexes.  
Most observations lie between 60 and 85 years, suggesting a concentrated and stable pattern across countries.

Figure~\ref{fig:male_female} visualises the relationship between male and female life expectancy.  
The points lie close to a straight line, confirming the extremely strong correlation between the two variables ($r \approx 0.99$).  
Almost all points are above the $y=x$ line, indicating that female life expectancy is nearly always higher.

Figure~\ref{fig:dist_gender} compares the distributions for male, female, and combined life expectancy.  
The curves overlap heavily and have almost identical shapes.  
The combined “both” distribution lies between male and female, consistent with the computed correlations (all $r > 0.99$).

\section{Discussion}

The numerical and visual analyses show that life expectancy behaves very similarly across genders.  
The extremely high correlations between male, female, and combined life expectancy imply that improvements in health conditions or societal factors affect the entire population uniformly.

Additionally, all three life expectancy measures show strong positive correlation with Year ($r \approx 0.75$).  
This indicates a clear global trend of increasing life expectancy over time, likely due to improvements in healthcare, hygiene, nutrition and education.

However, these findings describe associations, not causal relationships.  
Other socioeconomic factors may contribute to the observed patterns but are not included in this simple analysis.

\section{Conclusion}

The analysis reveals that:

\begin{itemize}
    \item Life expectancy for males, females, and both sexes is almost perfectly correlated.
    \item Female life expectancy is consistently higher than male life expectancy.
    \item Life expectancy has increased steadily over time.
\end{itemize}

These results help to understand global health patterns and provide a foundation for further studies using additional socioeconomic variables.

\end{document}